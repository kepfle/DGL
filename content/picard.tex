\chapter{Der Satz von Picard-Lindel\"of}
Wir betrachten das folgende
\[ (AWP)\begin{cases}
y'=f(x,y)\\y(\xi)=\eta
\end{cases} \]
und wollen in diesem Kapitel Bedingungen an $ f $ finden, die garantieren, dass $ (AWP) $ genau eine (lokale) L\"osung besitzt.
\begin{satz}[Picard-Lindel\"of]
	Seien $ \xi,\eta\in\R $, $ a,b>0 $. Sei $ Q=[\xi-a,\xi+a]\times[\eta-b,\eta+b] $. Die Funktion $ f\colon Q\rightarrow\R $ sei auf $ Q $ stetig und gen\"ugt dort einer Lipschitzbedingung  bez\"uglich $ y $, d.h.
	\[ \exists L\geq 0\forall (x,y_1), (x,y_2)\in Q: |f(x,y_1)-f(x,y_2)|\leq L|y_1-y_2|. \]
	Dann hat das $ (AWP) $ von oben im Intervall $ [\xi-h,\xi+h] $ mit $ h=\min\left\lbrace a,\frac{b}{m}\right\rbrace $, $ m=\max_{(x,y)\in Q}|f(x,y)| $, genau eine L\"osung.
\end{satz}
\begin{bemerkung}
	\begin{enumerate}
		\item[]
		\item Falls $ m=0 $, hei\ss t dies, dass $ f\equiv 0 $, also $ y\equiv\eta $ eine L\"osung auf $ [\xi-a,\xi+a] $. Wir k\"onnen also oben $ \frac{b}{0}=\infty $ lesen.
		\item Die Lipschitzbedingung ist eine Wachstumsbeschr\"ankung an $ f $ in $ y- $Richtung bei festgehaltenem $ x $, wobei die Lipschitzkonstante $ L $ unabh\"angig von $ x $ ist.
		\item Dass $ (\xi,\eta) $ in der Mitte von $ Q $ liegt, haben wir auf Bequemlichkeit angenommen. Wir diskutieren sp\"ater noch lokale und globale Lipschitzbedingungen sowie die Frage ob der Anfangswert auch auf dem Rand des Quaders liegen darf.\\
		Erstmal geben wir uns mit dem Quader zufrieden und bringen hierf\"ur eine hinreichende Bedingung sowie Beispiele.
	\end{enumerate}
\end{bemerkung}
\begin{lemma}
	Sei $ f\colon Q\rightarrow\R $ stetig, $ \frac{\partial f}{\partial y}\equiv f_y $ existiere und sei auf $ Q $ beschr\"ankt (also z.B. $ f_y $ stetig ist). Dann gen\"ugt $ f $ auf $ Q $ einer Lipschitzbedingung in $ y $.
\end{lemma}
\begin{beweis}
	Seien $ (x,y_1) $, $ (x,y_2)\in Q $, $ y_1<y_2 $. Dann ist $ f(x,\cdot)\coloneq[y_1,y_2]\rightarrow\R $ stetig und auf $ ]y_1,y_2[ $ differenzierbar. Nach dem Mittelwertsatz existiert ein $ \tilde y\in]y_1,y_2[ $ so dass\\$ |f(x,y_1)-f(y,y_2)|=|f_y(x,\tilde y)||y_1-y_2| $. Da $ f_y $ auf $ Q $ beschr\"ankt ist, existiert ein $ L\geq 0 $ so dass f\"ur alle $ (x,y)\in Q $ gilt: $ |f_y(x,y)|\leq L $. Hieraus folgt die Behauptung.
\end{beweis}
\begin{beispiel}
	\begin{enumerate}
		\item[]
		\item Sei $ f\colon\R\times\R\rightarrow\R $, $ f(x,y)=y^2 $. Dann erf\"ullt $ f $ auf jedem (kompakten(!)) Quader eine Lipschitzbedingung bez\"uglich $ y $, denn
		\[ |f(x,y_1)-f(x,y_2)|=|y_1^2-y_2^2|=|y_1+y_2||y_1-y_2| \]
		und
		\[ |y_1+y_2|\leq\sup_{\tilde y_1,\tilde y_2\in[\xi-a,a+\xi]}|\tilde y_1+\tilde y_2|\eqqcolon L<\infty \]
		erf\"ulllt also sogar eine Lipschitzbedingung auf $ [\xi-a,a+\xi]\times\R $, aber nicht auf $ \R\times\R $.
		\item Sei $ f\colon\R\times\R\rightarrow\R $, $ f(x,y)=3\sqrt[3]{y^2} $. Das
		\[ (AWP)\begin{cases}
		y'=f(x,y)\\y(0)=0
		\end{cases} \]
		hat die L\"osungen $ y\equiv 0 $ und $ y=x^3 $ auf $ \R $. Nach Satz 7.1 kann $ f $ also auf keinem Quader um $ (\xi,\eta)=(0,0) $ eine Lipschitzbedingung bez\"uglich $ y $ erf\"ullen.
 	\end{enumerate}
\end{beispiel}
\begin{beweis}[Beweis von Satz 1]
	Der Beweis hat 4 Schritte:
	\begin{enumerate}
		\item Umformulierung des AWPs in eine \"aquivalente Integralgleichung: Wir zeigen, dass f\"ur $ y\colon I\rightarrow\R $, $ I\subseteq\R $ Intervall, $ \xi\in I $, die folgenden zwei Bedingungen \"aquivalent sind:
		\begin{enumerate}
			\item $ y=y(x) $ l\"ost das AWP $ y'=f(x,y) $, $ y(\xi)=\eta $ auf $ I $.
			\item $ y\in C(I) $ und 
			\[ y(x)=\eta+\int_{\xi}^{x}f(t,y(t))\dd t\forall x\in I. \]
		\end{enumerate}
		\begin{description}
			\item[a)$ \Rightarrow $b):] $ y=y(x) $ L\"osung$ \Rightarrow y'(t)=f(t,y(t))$ f\"ur $ t\in I $, letztere Funktion ist stetig, $ y(\xi)=\eta $. Es folgt:
			\[ y(x)-\eta=\int_{\xi}^{x}f(t,y(t))\dd t\forall x\in I. \]
			$ y\in C(I) $ ist klar, da $ y $ ja sogar differenzierbar ist.
			\item[b)$ \Rightarrow $a):] Sei $ y\in C(I) $ und es gelte die Integralgleichung von oben. $ f(r,y(r)) $ ist stetig auf $ I $, und mit dem Hauptsatz der Differentialrechnung folgt, dass
			\[ x\mapsto\int_{\xi}^{x}f(t,y(t)) \dd t \]
			differenzierbar und eine Stammfunktion von $ f(x,y(x)) $ ist. Also $ y'(x)=f(x,y(x)) $ f\"ur $ x\in I $ und $ y(\xi)=\eta+0=\eta $. 
		\end{description}
		Wir k\"onnen also statt das AWP zu l\"osen, zeigen, dass genau eine Funktion $ y\in C[\xi-h,\xi+h] $ existiert mit
		\[ (IGL)\quad y(x)=\eta+\int_{\xi}^{x}f(t,y(t))\dd t \]
		f\"ur $ y\in[\xi-h,\xi+h] $.
		\item Umformulierung der IGL in ein Fixpunktproblem: Setze $ I\coloneqq[\xi-h,\xi+h] $,\\ $ M=\lbrace\varphi\in C[I]\mid\max_{t\in I}|\varphi(t)-\eta|\leq b\rbrace $ und \[ T\colon M\rightarrow M, \varphi\mapsto T(\varphi)\colon I\rightarrow\R, x\mapsto T(\varphi)(x)\coloneqq\eta+\int_{\xi}^{x}f(t,\varphi(t))\dd t \]
		Es gilt \[ T(\varphi)=\varphi\Leftrightarrow\forall x\in I:\varphi(x)=T(\varphi)(x)=\eta+\int_{\xi}^{x}f(t,\varphi(t))\dd t. \]
		Beachte: Wir wollen $ T(\varphi) $ wie oben definieren, damit wir (IGL) als Fixpunktproblem schreiben k\"onnen. Da $ f $ aber nicht auf $ I\times\R $, sondern nur auf $ Q $ definiert ist, k\"onnen wir nicht $ T\colon C(I)\rightarrow C(I) $ betrachten, sondern m\"ussen $ T $ auf $ M\subseteq C(I) $ definieren. So bekommen wir auf jeden Fall $ T\colon M\rightarrow C(I) $, dass $ T\varphi\in M $ gilt f\"ur jedes $ \varphi\in M $ m\"ussen wir noch pr\"ufen: F\"ur $ x\in I: $
		\[ |T(\varphi)(x)-\eta|=\left|\int_{\xi}^{x}f(t,\varphi(t))\dd t\right|\leq\int_{\xi}^{x}|f(t,\varphi(t))Y\dd t\leq\int_{\xi}^{x}\max_{(\tilde x,\tilde y)\in Q}|f(\tilde x,\tilde y)|\dd t=\max_{(\tilde x,\tilde y)\in Q}|f(\tilde x,\tilde y)||x-\xi|\leq m\cdot h\leq b \]
		Also:
		\[ \max_{x\in I}|T(\varphi)(x)-\eta|\leq b\Rightarrow T(\varphi)\in M \]
		\item Wahl einer Metrik auf $ M $, die $ T\colon M\rightarrow M $ zu einer Kontraktion und $ M $ vollst\"andig macht. Eine naheliegende Metrik auf $ M $ w\"are die folgende: Wir statten $ C(I) $ mit der Supremumsnorm aus. Damit ist $ C(I) $ ein Banachraum. $ M\subseteq(C(I),\norm{\cdot}_\infty) $ ist eine abgeschlossene Teilmenge und daher ein vollst\"andiger metrischer Raum bez\"uglich der induzierten Metrik. In dieser Metrik w\"urde $ T $ aber i.A. keine Kontraktion sein. Daher verwenden wir auf $ C(I) $ eine sogenannte \deftxt{Morgensternnorm}:
		\[ \norm{f}_{\infty,\alpha}=\max_{t\in I}e^{-\alpha|t-\xi|}|\varphi(t)| \]
		mit $ \alpha\in\R $. Das $ \alpha $ k\"onnen wir sp\"ater so w\"ahlen, dass wir eine Kontraktion bekommen. Zun\"achst aber die anderen Punkte:
		\begin{itemize}
			\item Aufg. 16$ \Rightarrow\norm{\cdot}_{\infty,\alpha}\sim\norm{\cdot}_\infty\Rightarrow (C(I),\norm{\cdot}_{\infty,\alpha}) $ vollst\"andig.
			\item $ M\subseteq(C(I),\norm{\cdot}_\infty) $ ist abgeschlossen, denn
			\[ M=\lbrace \varphi\in C(I)\mid\max_{t\in I}|\varphi(t)-\eta|\leq b\rbrace=\lbrace \varphi\in C(I)\mid\norm{\varphi-\psi}_{\infty}\leq b\rbrace=\bar B_b(\psi) \]
			\item Kapitel 5.16$ \Rightarrow M\subseteq (C(I),\norm{\cdot}_{\infty,\alpha}) $ abgeschlossen. Setze $ d_{\infty,\alpha}(\varphi,\psi)=\norm{\varphi-\psi}_{\infty,\alpha} $ f\"ur $ \varphi,\psi\in C(I) $ bzw. in $ M $.\\
	        Dann ist $ (M,d) $ vollst\"andig: Eine $ d_{\infty,\alpha}- $Cauchyfolge in $ M $ ist $ D_{\infty,\alpha}- $ bzw. $ \norm{\cdot}_{\infty,\alpha} -$Cauchyfolge in $ C(I) $, dort konvergent und wegen der Abgeschlossenheit liegt der Grenzwert in $ M $.
	        \item Nun behaupten wir, dass es ein $ \alpha\in\R $ gibt, so dass $ T\colon (M,d_{\infty,\alpha})\rightarrow (M,d_{\infty,\alpha}) $ kontraktierend ist .Also f\"ur $ \varphi,\psi\in M $:
	        \begin{align*}
	        d_{\infty,\alpha}(T(\varphi),T(\psi))&=\max_{x\in I}e^{-\alpha|x-\xi|}|T(\varphi)(x)-T(\psi)(x)|\\&=\max_{x\in I}e^{-\alpha|x-\xi|}\left|\int_{\xi}^{x}f(t,\varphi(t))-f(t,\psi(t))\dd t\right|\\&\leq\max_{x\in I}\left|\int_{\xi}^{x}|f(t,\varphi(t))-f(t,\psi(t))|\dd t\right|\\&\leq\max_{x\in I}e^{-\alpha|x-\xi|}L\left|\int_{\xi}^{x}|\varphi(t)-\psi(t)|e^{-\alpha|t-\xi|}e^{\alpha|t-\xi|}\dd t\right|\\
	        &\leq\max_{x\in I}e^{-\alpha|x-\xi|}L\left|\int_{\xi}^{x}e^{\alpha|t-\xi|}\dd t\right|d_{\infty,\alpha}(\varphi,\psi)\\
	        &\leq\frac{L}{\alpha}\max_{x\in I}e^{-\alpha|x-\xi|}(e^{\alpha|x-\xi|}-1)d_{\infty,\alpha}(\varphi,\psi)\\
	        &=\frac{L}{\alpha}\max_{x\in I}(1-e^{-\alpha|x-\xi|})d_{\infty,\alpha}(\varphi,\psi)\\
	        &\leq\frac{L}{\alpha}d_{\infty}(\varphi,\psi)
	        \end{align*}
	        Mit $ \alpha>L $ erhalten wir also, dass $ T $ eine Kontraktion ist.
	     \end{itemize}
	     \item Anwendung des Banachschen Fixpunktsatzes. Es gibt also genau ein $ y\in M $ mit $ T(y)=y $.
	\end{enumerate}
\end{beweis}