 \chapter{Lineare Differentialgleichungen h\"oherer Ordnung}
 \begin{definition}
 	Ist $ L $ wie in 3., so nennen wir $ y_1,...,y_n $ ein \deftxt{Fundamentalsystem}, falls $ y_1,...,y_n $ ein linear unabh\"angiges System von L\"osungen von $ Ly=0 $ ist, d.h. eine Basis des L\"osungsraums $ \ker L $ ist.
 \end{definition}
 Leider k\"onnen wir - Im Gegensatz zur ersten Ordnung - keine allgemeine L\"osungsformel finden. Aber wir k\"onnen, wenn wir eine L\"osung kennen, die Ordnung des Problems reduzieren.
 \begin{beispiel}
 	\[ x^2y''-xy'+y=0,\quad x>0 \]
 	D.h. \[ \tilde L=x^2\frac{\dd^2}{\dd x^2}-x\frac{\dd}{\dd x}1 \]
 	auf $ C^2]0,\infty[ $. Wir raten die L\"osung $ y_1(x)=x $ und machen den Ansatz
 	\[ y_2(x)=y_1\int v\dd x=x\int v\dd x \]
 	wobei wir jetzt $ v $ so zu bestimmen haben, dass $ y_2 $ die DGL $ \tilde Ly=0 $ l\"ost.\\
 	Wir bestimmen
 	\[ y'_2(x)=\int v\dd x+xv,\quad y_2''=v+v+xv'=2v+xv'. \]
 	\begin{align*} \tilde L y_2&=x^2y_2''-xy_2'+y_2\\&=x^2(2v+xv')-x\left(\int v\dd x+xv \right)+x\int v\dd x\\&=2x^2v+x^2v'-x\int v\dd x-x^2v+x\int v\dd x\\&=x^2v+x^3v'\\&=0 \end{align*}
 	Es folgt:
 	\[ v'=-\frac{1}{x}v\Rightarrow \ln|v|=-\ln|x|\Rightarrow v=e^{-\ln x}=\frac{1}{x}\rightarrow y_2(x)=x\int v\dd x=x\int\frac{1}{x}\dd x=x\ln x. \]
 	Probe:
 	\begin{align*}
 	&y_2'(x)=\ln x+x\frac{1}{x}=\ln x+1\\
 	&y_2''(x)=\frac{1}{x}\\
 	\Rightarrow&\tilde L y_2=x^2\frac{1}{x}-x(\ln x+1)+x\ln x=x-x\ln x-x+x\ln x=0
 	\end{align*}
 	\[ W(y_1(x),y_2(x))=\det \begin{pmatrix}
 	y_1(x)&y_2(x)\\y_1'(x)&y_2'(x)
 	\end{pmatrix}=\det \begin{pmatrix}
 	x&x\ln x\\1&\ln x+1
 	\end{pmatrix}=x(\ln x+1)-(x\ln x)1=x\ln x+x-x\ln x=x\neq 0 \]
 	f\"ur $ x\in ]0,\infty[ $, d.h. $ \lbrace y_1,y_2\rbrace $ linear unabh\"angig, also Fundamentalsystem.
 \end{beispiel}
 \begin{satz}
 	Sei $ L $ wie in 3., $ y_1=y_1(x) $ sei eine L\"osung von $ Ly=0 $, die $ \neq 0 $ auf $ I $ ist.
 	\begin{enumerate}
 		\item Dann ist
 		\[ \frac{1}{y_1(x)}L\left(y_1\int v(x)\dd x\right)=0\qquad(\ast) \]
 		eine lineare homogene DGL $ (n-1)- $ter Ordnung in $ v $.
 		\item Bilden $ v_2=v_2(x),...,v_n=v_n(x) $ ein Fundamentalsystem f\"ur $ (\ast) $, dann bilden \[ y_1(x),y_2(x)=y_1(x)\int v_2(x)\dd x,...,y_n(x)=y_1(x)\int v_n(x)\dd x \]
 		ein Fundamentalsystem f\"ur $ Ly=0 $.
 	\end{enumerate}
 \end{satz}
 \begin{beweis}
 	\begin{enumerate}
 		\item Setze $ \int v(x)\dd x\eqqcolon u $ und wende die Produktregel
 		\[ (y_1u)^{(k)}=\sum_{j=0}^{k}\binom{k}{j}y_1^{(k-j)}u^{(k)} \]
 		an um $ L(y_1u) $ zu berechnen:
 		\begin{align*}
 		&a_0y_1 u=a_0y_1u\\
 		&a_1(y_1u)'=a_1y_1u+a_1y_1u'\\
 		&a_2(y_1u)''=a_2y''u+2a_2y_1'u'+a_2y_1u''\\
 		&\vdots\\
 		&a_{n-1}(y_1u)^{(n-1)}=a_{n-1}y_1^{(n-1)}u+(n-1)a_{n-1}y_1^{(n-2)}u'+...+a_{n-1}y_1u^{(n-1)}\\
 		+&(y_1u)^{(n)}=y_1^{(n)}u+ny_1^{(n-1)}u'+...+y_1u^{(n)}\\
 		&\\
 		&L(y_1u)=(Ly_1)u+\alpha_1u'+...+\alpha_{n-1}u^{(n-1)}+y_1u^{(n)}\\
 		=&0+\alpha_1 v+...+\alpha_{n-1}v^{(n-2)}y_2 v^{(n-1)}
 		\end{align*}
 		Also:
 		\[ \frac{1}{y_1(x)}L\left(y_1\int v\dd x\right)=\frac{\alpha_1}{y_1(x)}v+...+\frac{\alpha_{n-1}}{y_1(x)}v^{(n-2)}+v^{(n-1)} \]
 		und somit ist $ (\ast) $ lineare homogene DGL $ (n-1)- $ter Ordnung in $ v $.
 		\item Ist $ v $ eine L\"osung von $ (\ast) $, so ist $ y\coloneqq y_1(x)\int v\dd x $ eine L\"osung von $ Ly=0 $. Seien $ v_1,...,v_n $ ein Fundamentalsystem f\"ur $ (\ast) $, d.h. linear unabh\"angig. Es muss gezeigt werden , dass die $ y_1,...,y_n $ dann ebenfalls linear unabh\"angig sind.\\
 		Gelte $ c_1y_1+...+c_ny_n=0 $ mit $ c_1,...,c_n\in\R $, also
 		\begin{align*} &c_1y_1+c_2y_1\int v_2(x)\dd x+...+c_ny_1\int v_n(x)\dd x=0\\\Rightarrow&c_1+c_2\int v_2(x)\dd x+...+c_n\int v_n(x)\dd x=0\\
 		\Rightarrow&c_2v_2+...+c_nv_n=0\\
 		\Rightarrow&c_2=...=c_n=0\Rightarrow c_1y_1=0\Rightarrow c_1=0. \end{align*}
 		D.h. $ \lbrace y_1,...,y_n\rbrace $ ist Fundamentalsystem. 
 	\end{enumerate}
 \end{beweis}
 Nehmen wir mal an, wir haben . z.B. durch Raten und Reduktion -  ein Fundamentalsystem $ y_!,...,y_n $ f\"ur $ Ly=0 $ gefunden. Jetzt wollen wir analog zur 1. Ordnung per Variation der Konstanten eine partikul\"are L\"osung von $ Ly=b $ finden.\\
 Ansatz:
 \[ y_p=\sum_{i=1}^{n}c_i(x)y_i\Rightarrow y_p'(x)=\sum_{i=0}^{n}c_1'(x)y_i(x)+\sum_{i=1}^{n}c_i(x)y_i'(x) \]
 Foderung: $ \sum c_i'y_i=0 $. Dann:
 \[ y_p''(x)=\sum_{i=1}^{n}c_i(x)y_i(x)+\sum_{i=1}^{n}c_i(x)y_i''(x) \]
 Forderung: $ \sum c_i'y_i'=0 $, usw.
 Am Ende:
 \[ y_p^{(n)}=\sum_{i=1}^{n}c_i'(x) y_i^{(n-1)}(x)+\sum_{i=1}^{n}c_i y_i^{(n)}(x) \]
 Sei nun $ \sum c_iy_i^{(n-1)}=b $. Damit erhalten wir das folgende LGS:
 \begin{align*}
 &y_1c_1'+y_2c_2'+...+y_nc_n'=0\\
 &y_1'c_1'+y_2'c_2'+...+y_n'c_n'=0\\
 &\vdots\\
 &y_1^{(n-1)}c_1'+y_2^{(n-1)}c_2'+...+y_n^{(n-1)}c_n'=b
 \end{align*}
 9. impliziert $ W(y_1(x),...,y_n(x))\neq 0 $ f\"ur alle $ x\in I $. Nach der Cramerschen Regel gilt nun
 \[ c_i'(x)=\frac{\det A_i(x)}{\det A(x)} \]
 mit $ A_1=A $, aber mit $ \begin{pmatrix}
 0\\\vdots\\0\\b
 \end{pmatrix} $ statt $ i- $ter Spalte. Obiges liefert f\"ur jedes $ x $ die eindeutige L\"osung und da $ A_i $, $ A $ stetig sind, ist $ \det A_i $, $ \det A $ stetig, d.h. $ c_i' $ stetig. Die gesuchten $ c_i $ findn wir nun per $ c_i=\int c_i'\dd x $.
 \begin{satz}
 	Sei $ L $ wie in 3.m $ b\colon I\rightarrow\R $ stetig. $ c_i=c_i(x) $ sei definiert wie oben f\"ur $ i=1,...,n $. Dann ist $ y_b(x)=\sum_{i=1}^{n}c_i(x) y_i(x) $ L\"osung von $ Ly=b $.
 \end{satz}
 \begin{beweis}
 	\begin{align*}
 	Ly_p&=y_p^{(n)}+a_{n-1}y_p^{(n-1)}+...+a_0y_p\\
 	&=b+\sum_{i=1}^{n}c_i(x) y_i^{(n)}+a_{n-1}\sum_{i=1}^{n}c_i(x)y_i^{(n-1)}+...+a_0\sum_{i=1}^{n}c_i(x)y_i\\
 	&=b+\sum_{i=1}^{n}c_i(x)\left(y_i^{(n)}+a_{n-1}y_i^{(n-1)}+...+a_0y_i\right)\\
 	&=b
 	\end{align*}
 \end{beweis}
 \begin{beispiel}
 	\[ y''-\frac{x}{x-1}y'+\frac{1}{x-1}y=e^x(x-1),\quad x>1 \]
 	Homogene DGL hat L\"osungen $ y_1=e^x $, $ y_2=x $ und diese sind offenbar linear unabh\"angig, d.h. $ \lbrace y_1,y_2\rbrace $ ist Fundamentalsystem.
 	\[ y_p(x)=_1(x)e^x+c_2(x) x \]
 	liefert
 	\begin{align*}
 	&y_1c_1'+y_2c_2'=0\\&y_1'c_1'+y_2'c_2'=e^x(x-1)
 	\end{align*}
 	Einsetzen:
 	\begin{align*}
 	&e^xc_1'+xc_2'=0\Rightarrow e^xc_1'=-xc_2'\\
 	&e^xc_1'+c_2'=e^x(x-1)\Rightarrow c_2'(1-x)=e^x(x-1)\Rightarrow c_2'=-e^x\Rightarrow c_1'=x
 	\end{align*}
 	Nach Integration: $ c_1=\frac{x^2}{2} $, $ c_2=-e^x $.
 	\[ y_p=e^x\left(\frac{x^2}{2}-x\right),\qquad y_{allg}^{inhom}=c_1e^x+c_2x+e^x\left(\frac{x^2}{2}-x\right),\quad c_1,c_2\in\R \]
 \end{beispiel}
 Als letztes in diesem Kapitel betrachten wir homogene lineare DGLn mit konstanten Koeffizienten, dieser Typ DGL erlaubt die explizite Angabe eines Fundamentalsystems. Also nur auf Intervall $ I $:
 \[ Ly=y^{(n)}+a_{n-1}y^{(n-1)}+...+y_0 y=0,\quad a_i\in\R \]
 und wir suchen $ n $ linear unabh\"angige L\"osungen.\\
 Ansatz: \begin{align*} y=e^{\lambda x}&\Rightarrow y'=\lambda e^{\lambda x}, y''\lambda^2 e^{\lambda x},..., y^{(n)}=\lambda^n e^{\lambda x}\\
 &\Rightarrow Ly=\lambda^n e^{\lambda x}+a_{n-1}\lambda^{n-1}e^{\lambda x}+...+a_0e^{\lambda x}=0\\
 &\Rightarrow p(\lambda)\coloneqq\lambda^n+a_{n-1}\lambda^{n-1}+...+a_0=0\qquad(\ast) \end{align*}
 $ (\ast) $ hei\ss t \deftxt{charakteristische Gleichung} der DGL $ Ly=0 $, es liegt eine algebraische Gleichung $ n- $ten Grades vor, d.h. in $ \C $ haben wir genau $ n $ L\"osungen, wenn wir Vielfachheiten mitz\"ahlen.
 \begin{enumerate}
 	\item $ (\ast) $ hat $ n $ paarweise verschiedene reelle L\"osungen $ \lambda_1,...,\lambda_n $. Dann sind $ y_1=e^{\lambda_1 x},...,y_n=e^{\lambda_n x} $ linear unabh\"angige L\"osungen.
 	\begin{align*} W(y_1,...,y_n)&=\det \begin{pmatrix}
 	e^{\lambda_1 x}&e^{\lambda_2 x}&\hdotsfor{1}&e^{\lambda_n x}\\
 	\lambda_1 e^{\lambda_1 x}&\lambda_2 e^{\lambda_2 x}&\hdotsfor{1}&\lambda_n e^{\lambda_n x}\\
 	\vdots\\
 	\lambda_1^{(n-1)}e^{\lambda_1 x}&\lambda_2^{(n-1)}e^{\lambda_2 x}&\hdotsfor{1}&\lambda_n^{(n-1)}e^{\lambda_n x}
 	\end{pmatrix}\\&=e^{\lambda_1 x}e^{\lambda_2 x}...e^{\lambda n x}\det \begin{pmatrix}
 	1&\hdotsfor{1}&1\\
 	\lambda_1&\hdotsfor{1}&\lambda_n\\
 	\vdots\\
 	\lambda_1^{(n-1)}&\hdotsfor{1}&\lambda_n^{(n-1)}
 	\end{pmatrix}\\&=e^{\left(\sum_{i=1}^{n}\lambda_i\right)x}\prod_{i<k}^{}(\lambda_i-\lambda_k)\neq 0 \end{align*}
 	da $ \lambda_i\neq\lambda_k $. Also ist $ \lbrace y_1,...,y_n\rbrace $ Fundamentalsystem.
 	\item $ (\ast) $ hat nur reelle L\"osungen, nicht notwendig von Vielfachheit 1.\\
 	Sei $ \lambda_0 $ eine L\"osung mit Viehcfachheut $ \nu_0>1 $. Dann sind $ y_1=e^{\lambda_0 x}, y_2=xe^{\lambda_0 x},...,a_{\nu_0}=x^{\nu_0-1}e^{\lambda_0 x} $ linear unabh\"angige L\"osungen. Schreibe
 	\[ p(\lambda)=(\lambda-\lambda_1)^{\nu_1}...(\lambda-\lambda_m)^{\nu_m},\quad \lambda_1\neq\lambda_j\text{ f\"ur }i\neq j, \nu_1+...+\nu_m=n \]
 	Partialbruchzerlegung:
 	\[ \frac{1}{p(\lambda)}=\frac{q_1(\lambda)}{(\lambda-\lambda_1)^{\nu_1}}+...+\frac{q_m(\lambda)}{(\lambda-\lambda_m)^{\nu_m}},\quad q_k\text{ geeignete Polynome} \]
 	\[ 1=q_1(\lambda)p_1(\lambda)+...+q_m(\lambda)p_m(\lambda),\quad p_k(\lambda)=\prod_{\substack{j=1\\j\neq k}}^{m}(\lambda-\lambda_j)^{\nu_j}\qquad (+) \]
 	\begin{bemerkung*}[Schreibweise]
 		Sei $ q(\mu)=b_k\mu^k+...+b_1\mu+b_0 $ ein Polynom. Wir k\"urzen $ u'=DU $, $ u''=D^2u $ ab und definieren den Differentialoperator
 		\[ q(D)\coloneqq b_kD^k+...+b_1 D+b_0 \]
 		oder ausf\"uhrlicher
 		\[ q(D)u\coloneqq b_ku^{(k)}+...+b_1u'+b_0u \]
 	\end{bemerkung*} 
 	Auf diese Weise gilt $ p(D)=L $, und aus $ (+) $ bekommen wir
 	\[ u=\underbrace{q_1(D)p_1(D)u}_{u_1}+...+\underbrace{q_m(D)p_m(D)u}_{=u_m}=u_1+...+u_m \]
 	f\"ur $ u\in C^\infty(I) $. Sei nun $ u $ eine L\"osung, d.h. $ p(D)u=0 $. Dann 
 	\begin{align*}
 	(D-\lambda_k)^{\nu_k}u_k&=(D-\lambda_k)^{\nu_k}q_k(D)p_k(D)u\\&=q_k(D)\underbrace{(D-\lambda_k)^{\nu_k}p_k(D)}_{=p(D)}u\\
 	&=q_k(D)p(D)u=0
 	\end{align*}
 	Jede L\"osung $ u $ von $ Ly=0 $ hat die Form $ u=u_1+...+u_m $ mit $ (D-\lambda_k)^{\nu_k}u_k=0 $ f\"ur $ k=1,...,m $. Umgekehrt gilt f\"ur $ u_k $ mit $ (D-\lambda_k)^{\nu_k}u_k=0 $:
 	\[ p(D)u_k=p_k(D)(D-\lambda_k)^{\nu_k}u_k=0 \]
 	Somit ist auch $ p(D)(u_1+...+u_m)=0 $.\\
 	Wir m\"ussen also die Teilgleichungen $ (D-\lambda_k)^{\nu_k}u_k=0 $ f\"ur $ k=1,...,m $ l\"osen. Aufgabe 29:
 	\[ u_k=(c_{k,0}+c_{k,1}x+...+c_{k,\nu_k-1}x^{\nu_k-1})e^{\lambda_{\nu_k}x} \]
 	f\"ur $ k=1,...,m $. Dass die Summanden linear unabh\"angig sind ist leicht zu sehen.
 	\item $ \lambda_0=\alpha+\beta i $ ist nicht-reelle L\"osung der Vielfachheit $ \nu_0>1 $. Dann ist $ \bar{\lambda}_0=\alpha-\beta i $ ebenfalls L\"osung der Vielfachheit $ \nu_0>1 $. Die Argumentation in ii) funktioniert auch komplex; wir erinnern uns, dass f\"ur $ f\colon I\rightarrow\C $, $ f'\coloneqq\Re f'+i(\Im f)' $ definiert ist. Also erhalten wir linear unabh\"angige komplexwertige L\"osungen $ e^{\lambda_0 x}, xe^{\lambda_0 x},..., x^{\nu_0-1}e^{\lambda_0 x} $ und per \[ e^{\lambda_0 x}=e^{\alpha+i\beta}=e^{\alpha x}e^{i\beta x}=e^{\alpha x}(\cos\beta x+i\sin\beta x) \]
 	\[ xe^{\lambda_i x}=xe^{\alpha x}(\cos\beta+i\sin\beta x),...,x^{\nu_0-1}(\cos\beta x+i\sin\beta x) \]
 	bekommen wir $ 2\nu_0 $ reelle L\"osungen
 	\[ y_{2k-1}=x^{k-1}e^{\alpha x}\cos\beta x,\quad y_{2k}=x^{k-1}e^{\alpha x}\sin\beta x,\quad k=1,...,\nu_0. \]
 \end{enumerate}
 \begin{bemerkung}
 	\begin{enumerate}
 		\item[]
 		\item Obige F\"alle k\"onnen zu einem Satz zusammengefasst werden, der zeigt wie f\"ur $ o\in\R[\lambda] $, $ L=p(D) $, stets ein Fundamentalsystem angegeben werden kann.
 		\item In konkreten Situationen ist das L\"osen der charakteristischen Gleichung nicht-trivial: bei h\"oheren Ordnungen evtl. praktisch unm\"oglich.
 		\item Wir bekommen, wenn wir das wollen, in i) immer L\"osungen auf $ I=\R $.
 		\item Den in i) erw\"ahnten Satz kann man (gehe i) bis iii) nochmal durch) auch f\"ur $ p\in\C[\lambda] $ bekommen. Beachte, dass wir nur komplexwertige DGLn dann betrachten und keine Ableitungen im Sinne der Funktionentheorie.
 	\end{enumerate}
 \end{bemerkung}
 \begin{beispiel}
 	\[ y^{(4)}+6y''+9y=0 \]
 	Charakteristische Gleichung aufstellen:
 	\[ p(\lambda)=\lambda^4+6\lambda^2+9=0\Leftrightarrow(\lambda^2+3)^2=0\Leftrightarrow\lambda=\pm isqrt{3},\text{ jeweils VFH }2 \]
 	L\"osungen sind also $ \lambda_0=\alpha+i\beta=0+i\sqrt{3} $, VFH 2, und $ \lambda_1=\alpha+i\beta=0+i\sqrt{3} $, VFH 2.\\
 	Fundamentalsystem:
 	\[ FS=\lbrace y_1=\cos\sqrt{3}x,y_2=\sin\sqrt{3}x,y_3=x\cos\sqrt{3}x,y_3=x\sin\sqrt{3}x\rbrace \]
 \end{beispiel}
