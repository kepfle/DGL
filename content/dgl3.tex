\chapter{Lineare Differentialgleichungen}
Wir betrachten eine DGL 1. Ordnung der Form
\[ (\ast)\qquad y'=a(x)y+b(x) \]
mit $ a,b\colon I\rightarrow\R $, $ I\subseteq\R $ ein Intervall.
\begin{definition}
	Die DGL hei\ss t \deftxt{lineare DGL 1. Ordnung}. Sie hei\ss t \deftxt{homogen}, wenn $ b=0 $ ist und \deftxt{inhomogen}, webb $ b\neq 0 $. In diesem Fall hei\ss t $ b $ die \deftxt{Inhomogenit\"at} oder \deftxt{St\"orfunktion}.
\end{definition}
\begin{beispiel}
	\begin{enumerate}
		\item[]
		\item Die Gleichung aus 1.1, \[ T'(t)=k(T_a-T(t))=\underbrace{-k}_{a(t)}T(t)\underbrace{-kT_A}_{b(t)}, \]
		ist eine inhomogene lineare DGL 1. Ordnung.
		\item Die DGL des exponentiellen Wachstums aus 1.3,
		\[ M'(t)=\underbrace{\alpha}_{a(t)}N(t), \]
		ist eine homogene lineare DGL 1. Ordnung.
		\item Die obige DGL modelliert den Wachstumsprozess ohne \"au\ss ere Einfl\"usse. Will man diese im Modell hinzuf\"ugen, kann man z.B. die Wachstumsrate $ \alpha $ zeitabh\"angig machen oder eine den Zuwachs \"anderende St\"orfunktion addieren. Dann erh\"alt man wieder eine inhomogene lineare DGL 1. Ordnung.  
	\end{enumerate}
\end{beispiel}